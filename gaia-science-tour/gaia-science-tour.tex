%%%%%%%%%%%%%%%%%%%%%%%%%%%%%%%%%%%%%%%%%%%%%%%%%%%%%%%%%%%%%%%%%%%%%%%%%%%%%%%%%%%%%%%%%%%%%%%%%
%
% TeX file gaia-science-tour.tex
% Uses: beamer.cls
% Last Updated:  2019.05.05
% First Created: 2019.04.28
%
% Title: Gaia science tour
%
% Description: Spitzer Lectures series 2019, Princeton University, May 2019
%
%%%%%%%%%%%%%%%%%%%%%%%%%%%%%%%%%%%%%%%%%%%%%%%%%%%%%%%%%%%%%%%%%%%%%%%%%%%%%%%%%%%%%%%%%%%%%%%%%

\documentclass[smaller, aspectratio=169]{beamer}

\usepackage{times,amsmath,graphicx,marvosym}
\usepackage{tikz}
\usepackage{animate}
\usepackage{colortbl}
\usetikzlibrary{arrows.meta,shapes,calc,shadows,backgrounds}
\usetheme{lightbare169}

\hypersetup{pdftitle={Gaia science tour},
  pdfsubject={Spitzer Lectures series 2019, Princeton University, May 2019}, colorlinks=true,
pdfauthor={Anthony Brown}}

\tikzstyle{flow}=[-{Stealth[round]}, thick, shorten >=3pt, shorten <=3pt]
\tikzstyle{flowboth}=[{Stealth[round]}-{Stealth[round]}, thick, shorten >=3pt, shorten <=3pt]

\setbeamercovered{invisible}

\graphicspath{ {./Images/} {/home/brown/Gaia/Presentation/Images/} }

\newcommand\gdrone{Gaia~DR1}
\newcommand\gdrtwo{Gaia~DR2}
\newcommand\hip{Hipparcos}
\newcommand\tyctwo{Tycho-2}
\newcommand\tyc{Tycho}
\input{/home/brown/Gaia/Presentation/GaiaDR2OverviewSlides/dr2stats}

\newcommand\llslides{\href{https://www.cosmos.esa.int/documents/29201/1770596/Lindegren_GaiaDR2_Astrometry_extended.pdf/1ebddb25-f010-6437-cb14-0e360e2d9f09}{Lindegren
et al.\ slide set}}

\title[Spitzer Lectures May 2019]{Gaia science tour}
\author{Anthony Brown}
\institute{Leiden Observatory, Leiden University\\\texttt{brown@strw.leidenuniv.nl}}

\begin{document}
\logos{
}

%\begin{frame}
%  \titlepage
%\end{frame}

\setbeamercolor{background canvas}{bg=black}
\begin{emptyframe}{Title page}
  \hglue-0.57truecm
  \begin{tikzpicture}
    \node (fig) at (current page)
    {\includegraphics[width=15.9cm]{GaiaSky/GaiaDR2/ESA-PR/Gaia_s_sky_in_colour.jpg}};
      \node at ($(fig.south east)+(-0.3,0)$) [anchor=north east, font=\sf\tiny, color=white] {ESA/Gaia/DPAC};
      \node at ($(fig.south west)+(0.5,2.5)$) [anchor=south west, rotate=-45]
      {\includegraphics[height=1.8cm]{Promotion/ArtistsImpression2013.png}};
      \node (title) at ($(current page.north)+(0,-1.0)$) [anchor=north, text width=\textwidth, font=\Huge, text badly centered] {
        \structure{\color{GaiaRed}\inserttitle}
      };
      \node (author) at ($(title.south)+(0,-0.5)$) [anchor=north, text width=\textwidth, font=\Large,
      text badly centered, color=white] {\insertauthor};
      \node (institute) at (author.south) [anchor=north, text width=\textwidth, font=\large, text
      badly centered, color=white] {\insertinstitute};
  \end{tikzpicture}
\end{emptyframe}
%
%%%%%%%%%%%%%%%%%%%%%%%%%%%%%%%%%%%%%%%%%%%%%%%%%%%%%%%%%%%%%%%%%%%%%%%%%%%%%
%
\begin{emptyframe}{Gaia covers all cosmic scales}
  \hglue-0.57truecm
  \begin{tikzpicture}
    \node (fig) at (current page)
    {\includegraphics[width=15.9cm]{GaiaDR2/ESA_Gaia_DR2_CosmicScales_Science_Infographic.jpg}};
    \node at ($(fig.north west)+(4,-0.2)$) [anchor=north west, font=\tiny, color=white]
    {Credits: ESA};
  \end{tikzpicture}
\end{emptyframe}
\setbeamercolor{background canvas}{bg=white}
%
%%%%%%%%%%%%%%%%%%%%%%%%%%%%%%%%%%%%%%%%%%%%%%%%%%%%%%%%%%%%%%%%%%%%%%%%%%%%%
%
\begin{agaframe}{Impact of Gaia DR1 and DR2}
  \underbar{Gaia DR1}
  \begin{itemize}
    \item from $10^5$ to 2 million high precision parallaxes and proper motions
    \item accurate all-sky positional reference frame to $G=20.7$
    \item all-sky homogeneous white light photometry to $G=20.7$
  \end{itemize}

  \medskip
  \underbar{Gaia DR2}
  \begin{itemize}
    \itemd $1.3$ billion high precision, all-sky, homogeneous, parallaxes and proper motions to
      $G=20.7$
      \begin{itemize}
        \item precise geometric distances over several kpc around the Sun
      \end{itemize}
    \itemd high precision, homogeneous, all-sky photometric survey to $G=20.7$, three broad bands
    \itemd First realization of optical reference frame from extragalactic sources
    \itemd Large, all-sky, homogeneous, and precise radial velocity survey
  \end{itemize}

  \medskip
  \underbar{We are getting used to}
  \begin{itemize}
    \item Looking up a parallax/ proper motion/ colour for any source in the sky
    \item Much more opportunity for stellar occultation campaigns based on very precise predictions
    \item Calibrating existing and future surveys onto Gaia astrometry/photometry
      \begin{itemize}
        \item including surveys from 100 years ago
      \end{itemize}
  \end{itemize}

  \medskip
  Following literature survey is a biased and very incomplete sample from the around 2000 papers
  using Gaia data since September 2016.
\end{agaframe}
%
\input{/home/brown/Gaia/Presentation/GaiaTalkBuildingBlocks/GaiaDR1Science/plutogaiadr0}
%
\input{/home/brown/Gaia/Presentation/GaiaTalkBuildingBlocks/GaiaDR1Science/hsoy}
%
\input{/home/brown/Gaia/Presentation/GaiaTalkBuildingBlocks/GaiaDR1Science/lmcrotation}
%
\input{/home/brown/Gaia/Presentation/GaiaTalkBuildingBlocks/GaiaDR2Science/lmc-smc-kinematics}
%
\input{/home/brown/Gaia/Presentation/GaiaTalkBuildingBlocks/GaiaDR1Science/lmc-smc-structure}
%
\input{/home/brown/Gaia/Presentation/GaiaTalkBuildingBlocks/GaiaDR2Science/cloudsinarms}
%
\input{/home/brown/Gaia/Presentation/GaiaTalkBuildingBlocks/GaiaDR1Science/sculptordr1}
%
\input{/home/brown/Gaia/Presentation/GaiaTalkBuildingBlocks/GaiaDR2Science/m31-m33}
%
\input{/home/brown/Gaia/Presentation/GaiaTalkBuildingBlocks/GaiaDR1Science/hvsdr1}
%
\input{/home/brown/Gaia/Presentation/GaiaTalkBuildingBlocks/GaiaDR2Science/hvsdr2}
%
\input{/home/brown/Gaia/Presentation/GaiaTalkBuildingBlocks/GaiaDR1Science/gaia1cluster}
%
\input{/home/brown/Gaia/Presentation/GaiaTalkBuildingBlocks/GaiaDR1Science/cmdanderson}
%
\input{/home/brown/Gaia/Presentation/GaiaTalkBuildingBlocks/GaiaDR1Science/oriondr1}
%
\input{/home/brown/Gaia/Presentation/GaiaTalkBuildingBlocks/GaiaDR2Science/micro-lensing-predictions}
%
\input{/home/brown/Gaia/Presentation/GaiaTalkBuildingBlocks/GaiaDR2Science/lensed-qsos}
%
\input{/home/brown/Gaia/Presentation/GaiaTalkBuildingBlocks/GaiaDR2Science/exoplanet-protoplanetarydisk-radii.tex}
%
\end{document}

%%%%%%%%%%%%%%%%%%%%%%%%%%%%%%%%%%%%%%%%%%%%%%%%%%%%%%%%%%%%%%%%%%%%%%%%%%%%%%%%%%%%%%%%%%%%%%%%%
%
% TeX file gaia-dataproducts-and-beyondgaia.tex
% Uses: beamer.cls
% Last Updated:  2019.05.02
% First Created: 2019.04.19
%
% Title: Gaia: Celestial Inventory from the Solar System to the Milky Way Neighbourhood
%
% Description: Spitzer Lectures series 2019, Princeton University, May 2019
%
%%%%%%%%%%%%%%%%%%%%%%%%%%%%%%%%%%%%%%%%%%%%%%%%%%%%%%%%%%%%%%%%%%%%%%%%%%%%%%%%%%%%%%%%%%%%%%%%%

\documentclass[smaller, aspectratio=169]{beamer}

\usepackage{times,amsmath,graphicx,marvosym}
\usepackage{tikz}
\usepackage{animate}
\usepackage{colortbl}
\usetikzlibrary{arrows.meta,shapes,calc,shadows,backgrounds}
\usetheme{lightbare169}

\hypersetup{pdftitle={Gaia: Celestial Inventory from the Solar System to the Milky Way Neighbourhood},
pdfsubject={Spitzer Lectures series 2019, Princeton University, May 2019},
pdfauthor={Anthony Brown}, colorlinks=true, linkbordercolor={1 0 0}}

\tikzstyle{flow}=[-{Stealth[round]}, thick, shorten >=3pt, shorten <=3pt]
\tikzstyle{flowboth}=[{Stealth[round]}-{Stealth[round]}, thick, shorten >=3pt, shorten <=3pt]

\setbeamercovered{invisible}

\graphicspath{ {./Images/} {/home/brown/Gaia/Presentation/Images/} }

\newcommand\gdrone{Gaia~DR1}
\newcommand\gdrtwo{Gaia~DR2}
\newcommand\hip{Hipparcos}
\newcommand\tyctwo{Tycho-2}
\newcommand\tyc{Tycho}
\input{/home/brown/Gaia/Presentation/GaiaDR2OverviewSlides/dr2stats}

\title[Spitzer Lectures May 2019]{Gaia: Celestial Inventory from the Solar System to the Milky Way}
\author{Anthony Brown}
\institute{Leiden Observatory, Leiden University\\\texttt{brown@strw.leidenuniv.nl}}

\begin{document}
\logos{
}

%\begin{frame}
%  \titlepage
%\end{frame}

\setbeamercolor{background canvas}{bg=black}
\begin{emptyframe}{Title page}
  \hglue-0.57truecm
  \begin{tikzpicture}
    \node (fig) at (current page)
    {\includegraphics[width=15.9cm]{GaiaSky/GaiaDR2/ESA-PR/Gaia_s_sky_in_colour.jpg}};
      \node at ($(fig.south east)+(-0.3,0)$) [anchor=north east, font=\sf\tiny, color=white] {ESA/Gaia/DPAC};
      \node at ($(fig.south west)+(0.5,2.5)$) [anchor=south west, rotate=-45]
      {\includegraphics[height=1.8cm]{Promotion/ArtistsImpression2013.png}};
      \node (title) at ($(current page.north)+(0,-1.0)$) [anchor=north, text width=\textwidth, font=\Huge, text badly centered] {
        \structure{\color{GaiaRed}\inserttitle}
      };
      \node (author) at ($(title.south)+(0,-0.5)$) [anchor=north, text width=\textwidth, font=\Large,
      text badly centered, color=white] {\insertauthor};
      \node (institute) at (author.south) [anchor=north, text width=\textwidth, font=\large, text
      badly centered, color=white] {\insertinstitute};
  \end{tikzpicture}
\end{emptyframe}
\setbeamercolor{background canvas}{bg=white}
%
%%%%%%%%%%%%%%%%%%%%%%%%%%%%%%%%%%%%%%%%%%%%%%%%%%%%%%%%%%%%%%%%%%%%%%%%%%%%%
%
\begin{agaframe}{About these lectures}
  \begin{itemize}
    \item All slides and other materials on Github
      \begin{itemize}
        \item \url{https://github.com/agabrown/spitzer-lectures-may2019}
      \end{itemize}
    \item Hyperlinks to papers and web pages are `clickable'
    \item Feel free to re-use the slides but please credit appropriately
      \begin{itemize}
        \item i.e., use original references to the figures
        \item unless otherwise stated figure credits are `Anthony G.A.~Brown'
      \end{itemize}
  \end{itemize}
\end{agaframe}
%
\section[Gaia]{Gaia mission summary}
%
\input{/home/brown/Gaia/Presentation/GaiaTalkBuildingBlocks/Gaia/gaia-overview}
%
\input{/home/brown/Gaia/Presentation/GaiaTalkBuildingBlocks/Gaia/sifobs}
%
\input{/home/brown/Gaia/Presentation/GaiaTalkBuildingBlocks/Gaia/data-collection}
%
\input{/home/brown/Gaia/Presentation/GaiaTalkBuildingBlocks/alerts}
%
%%%%%%%%%%%%%%%%%%%%%%%%%%%%%%%%%%%%%%%%%%%%%%%%%%%%%%%%%%%%%%%%%%%%%%%%%%%%%
%
\setbeamercolor{background canvas}{bg=black}
\begin{emptyframe}{Where's Gaia?}
  \begin{tikzpicture}
    \node (fig) {\includegraphics[height=7cm]{GBOT/Gaia_seen_by_ESO_VST_eso1908a_1280.jpg}};
    \node at ($(fig.north west)+(-0.5,0.5)$) [color=GaiaRed, font=\large]{\sf Where's Gaia?};
    \node at (fig.south) [anchor=north, font=\scriptsize, text width=15cm] {
      \url{https://www.eso.org/public/news/eso1908/}\\
      \url{http://sci.esa.int/gaia/61327-observing-gaia-from-earth-to-improve-its-star-maps/}
    };
  \end{tikzpicture}
\end{emptyframe}
\setbeamercolor{background canvas}{bg=white}
%
%
\section[Data processing]{Data processing}
%
\input{/home/brown/Gaia/Presentation/GaiaTalkBuildingBlocks/DataProcessing/data-processing-inbrief}
%
\input{/home/brown/Gaia/Presentation/GaiaTalkBuildingBlocks/DataProcessing/teamwork}
%
\section[Data products]{Gaia data products (current and planned)}
%
\input{/home/brown/Gaia/Presentation/GaiaTalkBuildingBlocks/DataProcessing/data-products}
%
\section[Long term]{Mission extension and beyond Gaia}
%
%\input{/home/brown/Gaia/Presentation/GaiaTalkBuildingBlocks/dr3dr4}
%
%%%%%%%%%%%%%%%%%%%%%%%%%%%%%%%%%%%%%%%%%%%%%%%%%%%%%%%%%%%%%%%%%%%%%%%%%%%%%
%
\begin{emptyframe}{Gaia mission extension}
  \centering
  \textcolor{GaiaRed}{\Huge Gaia mission extension}
\end{emptyframe}
%
\input{/home/brown/Gaia/Presentation/GaiaTalkBuildingBlocks/Extension/gaia-extension-princeton2019}
%
%%%%%%%%%%%%%%%%%%%%%%%%%%%%%%%%%%%%%%%%%%%%%%%%%%%%%%%%%%%%%%%%%%%%%%%%%%%%%
%
\begin{emptyframe}{Beyond Gaia}
  \centering
  \textcolor{GaiaRed}{\Huge Beyond Gaia}
\end{emptyframe}
\input{/home/brown/Gaia/Presentation/GaiaTalkBuildingBlocks/BeyondGaia/space-astrometry-beyond-gaia}
%
\section[Conclusions]{Conclusions}
%
\input{/home/brown/Gaia/Presentation/GaiaTalkBuildingBlocks/literature-entrypoints}
%
\end{document}
